\lecture{8. I Will Make a New Covenant}{08}

%------------------------------------------------------------------------------
\section{Introduction}
% this to put in this one:
% There is NO separation between church and personal life.
%--------------------------------------
\begin{frame}{IntroSlide}
\begin{center}
	\includegraphics[draft, width=0.8\textwidth]{figures/dummy.png}
\end{center}

\note{09:30}
\end{frame}

%--------------------------------------
\begin{frame}{God wants a relationship with people}
\framesubtitle{Jeremiah 31:31-34}
	\keyversehiglight{I will make a new covenant}
	
\note{09:33}
\end{frame}

%--------------------------------------
\begin{goals}
\goal Understand what the New Covenant is
\goal Examine why the New Covenant was needed
\goal Assess how God's relationship with His people changed from the Old to the New Covenants

\note{09:36}
\note[item]{A little different this week.  We'll look at 5 scriptures and think about what the New Covenants is, why it was needed, and what changed from the Old Covenant to the New Covenant.}
\end{goals}

%------------------------------------------------------------------------------
\section{What is the New Covenant?}

%--------------------------------------
\begin{frame}{The New Covenant vs. the Old}
\framesubtitle{Jer. 31:31-34}

\begin{description}[from the least to the greatest]
\item[I will put my law within them] genuine obedience
\item[I will write it on their hearts] inward commitment
\item[I will be their God] a new focus
\item[they shall be my people] it's no longer physical Israel
\item[\ldots they shall all know me] faith not genealogy
\item[from the least to the greatest] every Christian's a priest
\item[I will forgive their iniquities] more than overlooking sins
\end{description}

\note{09:36}
\note[item]{A prep for the last 5 lessons.}
\note[item]{law within them -- Christians focus on commitment from the heart not outward rituals (\emph{e.g.}, Pharisees,`whitewashed tombs')}
\note[item]{write it on hearts -- not tablets.  Israelites should have done this (law `when you rise up and lay down').  Christians will get it right.}
\note[item]{I will be their God -- Christians are God's new focus}
\note[item]{They shall be my people -- no longer physical Israel}
\note[item]{\ldots they shall all know me -- from 'no longer teach\ldots' forward says you don't have to teach a Christian how to be saved.  All Christians already `know' God. Not true in the OT. You could be born an Israelite but grow up not knowing the Lord.}
\note[item]{least to greatest -- no distinction between clergy/laity, jews/gentiles, rich/poor, etc.}
\note[item]{forgive their iniquity -- You'll need a better sacrifice for this.}
\end{frame}

%--------------------------------------
\begin{frame}{The New Covenant according to Jesus}
\framesubtitle{Luke 22:14-23}

\begin{columns}[c]
\begin{column}{0.4\textwidth}
	\includegraphics[width=\columnwidth]{figures/communion.jpg}
\end{column}
\begin{column}{0.6\textwidth}
	\begin{itemize}
		\item New Covenant = Kingdom of God
		\item Payment made with Jesus' body.
		\item Signed with Jesus' blood
	\end{itemize}
\end{column}
\end{columns}

\note{09:38}
\note[item]{Jesus talks about both the kingdom and the new covenant being associated with his sacrifice}
\note[item]{Two part of a contract are mentioned here.}
\note[item]{The first is payment for services rendered.}
\note[item]{Note the Jesus says that \emph{he} is paying for \emph{our} side of the contract.}
\note[item]{Second, there's a signature, which really is ratification of the contract.}
\note[item]{When Jesus dies the New Covenant is ratified.}
\note[item]{\emph{If someone says ratification is Pentecost, then I'd say Pentecost is the date of effect.  It's ratified before, but the start date is 50 days later. But, we'll not take the analogy too far.}}
\end{frame}

%--------------------------------------
\begin{frame}{The New Covenant according to Ephesians}
\framesubtitle{Eph. 2}

\begin{columns}[c]
\begin{column}{0.4\textwidth}
	\includegraphics[draft, width=\columnwidth]{figures/dummy.png}
\end{column}
\begin{column}{0.6\textwidth}
	\begin{itemize}
		\item A resurrection (1-4)
		\item By grace through faith 
		\item The end of the Old Covenant (15)
		\item God's Holy Sanctuary (19-21)
	\end{itemize}
\end{column}
\end{columns}

\note{09:38}
\note[item]{Dead in sins - Fulfills Jer. 31:34.}
\note[item]{New Covenant people will be `resurrected', in more ways than one.}
\note[item]{The end of the Old Covenant - New Covenant is not based on law}
\note[item]{Holy Sanctuary - Both house (v 19) and tabernacle (v 21)}
\note[item]{Jesus is the cornerstone}
\note[item]{Apostles and prophets build on it}
\end{frame}

%--------------------------------------
\begin{frame}{The New Covenant according to Hebrews}
\framesubtitle{Heb. 10}

\begin{columns}[c]
\begin{column}{0.4\textwidth}
	\includegraphics[draft, width=\columnwidth]{figures/dummy.png}
\end{column}
\begin{column}{0.6\textwidth}
	\begin{itemize}
		\item Jesus' sacrifice is once for all (10)
		\item Jesus' sacrifice sanctifies forever (12)
		\item Jesus' enemies not defeated\ldots\emph{yet} (13)
	\end{itemize}
\end{column}
\end{columns}

\note{09:38}
\note[item]{There's only going to be one sacrifice in the New Covenant.}
\note[item]{That sacrifice is going to work for everyone for all time.}
\note[item]{There are pieces of the New Covenant that have yet to be fulfilled}
\note[item]{For instance, judging of the wicked, who by their actions reject the sacrifice of Jesus.}
\end{frame}

%--------------------------------------
\begin{frame}{The New Covenant according to II Corinthians}
\framesubtitle{II Cor. 3:7-18}

\begin{columns}[c]
\begin{column}{0.6\textwidth}
	\includegraphics[width=\columnwidth]{figures/freedom.jpg}
\end{column}
\begin{column}{0.4\textwidth}
	\begin{itemize}
		\item glory
		\item freedom
	\end{itemize}
\end{column}
\end{columns}

\note{09:38}
\note[item]{Neil already went over this, so we'll just review}
\end{frame}
%------------------------------------------------------------------------------
\section{It was needed because\ldots}

%--------------------------------------
\begin{frame}{We were dead with no chance for life}
\framesubtitle{Heb. 10}

\begin{columns}[c]
\begin{column}{0.4\textwidth}
	\includegraphics[draft, width=\columnwidth]{figures/dummy.png}
\end{column}
\begin{column}{0.6\textwidth}
	\begin{itemize}
		\item We were dead in sins (1-9)
		\item We are Gentiles (11-21)
	\end{itemize}
\end{column}
\end{columns}

\note{09:38}
\note[item]{There's only going to be one sacrifice in the New Covenant.}
\note[item]{That sacrifice is going to work for everyone for all time.}
\note[item]{There are pieces of the New Covenant that have yet to be fulfilled}
\note[item]{For instance, judging of the wicked, who by their actions reject the sacrifice of Jesus.}
\end{frame}


%------------------------------------------------------------------------------
\section{Mercy on all}

%------------------------------------------------------------------------------
\section{Once for all}

%------------------------------------------------------------------------------
\section{From glory to glory}

%------------------------------------------------------------------------------
\section{Review}

\begin{frame}{My covenant they broke}
	\begin{itemize}
		\item Goals
	\end{itemize}
	
\note{10:12}
\end{frame}

