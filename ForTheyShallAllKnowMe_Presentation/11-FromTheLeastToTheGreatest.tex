\lecture{11. From The Least To The Greatest}{11}

%------------------------------------------------------------------------------
\section{Introduction}

%--------------------------------------
\begin{frame}{An English breakfast}

\begin{columns}[c]
\begin{column}{0.65\textwidth}
	\includegraphics[width=\columnwidth]{figures/englishBreakfast.jpg}
\end{column}
\begin{column}{0.35\textwidth}
	\includegraphics[width=\columnwidth]{figures/clottedCream.jpg}
\end{column}
\end{columns}

\note{09:30}
\end{frame}

%--------------------------------------
\begin{frame}{God wants a relationship with people}
\framesubtitle{Jeremiah 31:31-34}
	\keyversehiglight{from the least of them to the greatest}
\note{09:34}
\end{frame}

%--------------------------------------
\begin{goals}
\goal Review the separation that existing under the Old Law between God and His people 
\goal Appreciate the breadth of people that make up the New Testament church
\goal Propose godly ways we can enhance the inclusiveness of the Lord's church

\note{09:35}
\end{goals}

%------------------------------------------------------------------------------
\section{Old vs. New}

%--------------------------------------
\begin{frame}{The tabernacle separated the Israelites from God}
\framesubtitle{Heb. 9:1-14}

\begin{itemize}
	\item The presence of God was in the Most Holy Place
	\item Even the high priest went in only once per year
	\item Gifts and sacrifices were offered that could not clean the conscience of the worshipers
	\item Regular people could never go in.
\end{itemize}

\note{09:38}
\end{frame}

%--------------------------------------
\begin{frame}{Jesus opened the way}
\framesubtitle{Heb. 10:19-22}

\begin{itemize}
	\item We now have confidence to enter the most holy place.
	\item We should draw near to God with a full assurance of faith
\end{itemize}

\note{09:40}
\end{frame}

%--------------------------------------
\begin{frame}{Jesus opened the way for everyone}
\framesubtitle{Eph. 3:6}

\begin{itemize}
	\item The mystery of the gospel is that Gentiles can be God's people.
	\item We are all\ldots
	\begin{itemize}
		\item fellow heirs
		\item members of the same body
		\item partakers of the promise
	\end{itemize}
	\item The church is (and should be) made up of all sorts of people
\end{itemize}

\note{09:44}
\end{frame}

%------------------------------------------------------------------------------
\section{The breadth of church}

%--------------------------------------
\begin{frame}{Early or Late}
\framesubtitle{Matt. 20:1-16}

\begin{itemize}
	\item Every Christian gets heaven as their reward, whether they obeyed the gospel early in life, or later.
	\item Avoid saying, ``Why can't they just get it together?''
	\item You don't want a congregation made entirely of `mature' Christians.
\end{itemize}

\note{09:48}
\note[item]{It's difficult for \emph{everyone} to change their bad habits.}
\note[item]{Certainly, we shouldn't let people languish in spiritual infancy (\emph{e.g.}, Hebrews 5:13.}
\note[item]{However, change can be slow}
\note[item]{If you don't have New Christians, watch out.}
\note[item]{Churches without new Christians are about to die.}
\end{frame}

%--------------------------------------
\begin{frame}{Rich or Poor}
\framesubtitle{James 2:1-7}

\begin{itemize}
	\item Don't favor people just because they appear wealthy.
	\item Don't avoid people who are dressed poorly.
\end{itemize}

\note{09:52}
\note[item]{Typically a rich person (at least on the outside) is going to have less to deal with than a poor person.}
\note[item]{A poor person is going to be more work.}
\note[item]{If you \emph{are} rich, don't expect the power you have in secular life to translate into the church.}
\note[item]{This can happen to anyone who is in charge of folks at their job.}
\note[item]{You get used to people following your orders, and you (consciously or not) think that should translate to church.}
\end{frame}

%--------------------------------------
\begin{frame}{Educated or uneducated}
\framesubtitle{I Cor. 1:26-28}

\begin{itemize}
	\item Not many Corinthian Christians were from upper social classes. (But, the were in Philippi)
	\item True Christianity is never going to be followed by the majority.
	\item Our appeal and power lies in spiritual authority not worldy wisdom.
\end{itemize}

\note{09:56}
\note[item]{Corinthians not many wise.}
\note[item]{We're pretty distrustful of education at Walnut Street}
\note[item]{The point Paul is making is that you don't try to beat intellectuals at their own game}
\note[item]{What we provide is never going to `make sense' to those who rely solely on their own intellect.}
\end{frame}

%------------------------------------------------------------------------------
\section{Build inclusiveness}

%--------------------------------------
\begin{frame}{Christians are each part of a body, and work toward unity.}
\framesubtitle{Eph. 4:1-16}

\begin{itemize}
	\item We share one body, one Spirit, one hope, one Lord, one faith, one baptism, one God
	\item What unites us is greater than what could divide us.
	\item God has given Christians to be apostles, prophets, missionaries, pastors, teachers.
	\item What else?
\end{itemize}

\note{10:00}
\note[item]{There are specific jobs for the Lord that Paul lists.}
\note[item]{Some of these, like apostles, aren't possible today.}
\note[item]{Some required spiritual gifts.}
\note[item]{Does that mean that God does not still give through his providence different and important abilities to the Christians in the church?}
\note[item]{Of course not!  People have and can develop talents in many areas that will benefit the Lord's work.}

\end{frame}

%--------------------------------------
\begin{frame}{Don't allow culture to separate you}
\framesubtitle{James 2:1-7}

\begin{itemize}
	\item Don't offend others.
	\item Don't be easily offended.
	\item Resist the urge to be friends with only those `like' you.
	\item Try to recognize cultural traditions vs. Biblical authority.
\end{itemize}

\note{10:04}
\note[item]{Traditions aren't bad, as long as, in the back of our mind, we recognize that they are traditions.}
\end{frame}

%--------------------------------------
\begin{frame}{What areas could we improve on to increase the inclusiveness of the Lord's church?}
\framesubtitle{}

\begin{itemize}
	\item Talk with everyone.
	\item Go to group meetings.
	\item Have people over to your house.
	\item \ldots
\end{itemize}

\note{10:08}
\end{frame}

%------------------------------------------------------------------------------
\section{Review}

\begin{frame}{From the Least to the Greatest}
	God has removed His separation from us.
	\begin{itemize}
		\item Regular people could never go into the physical tabernacle.
		\item Now \emph{everyone} can draw near to God through Jesus.
		\item The church is made up of all sort of people.	
	\end{itemize}
	Embrace the cultural breadth of the New Testament church.
	\begin{itemize}
		\item Heaven is our common reward.
		\item Don't favor people just because they appear wealthy.
		\item Our power lies in spiritual authority not worldy wisdom.
	\end{itemize}
	Find godly ways to enhance our inclusiveness
	\begin{itemize}
		\item Unity in the gospel is greater than cultural differences.
		\item Don't offend or be easily offended.
		\item Do things with Christians outside of worship.
	\end{itemize}
	
\note{10:12}
\end{frame}
