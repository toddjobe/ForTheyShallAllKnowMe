\chapter{Introduction}

\begin{goals}
\goal Become familiar with the class theme, key passage, and class objectives.
\goal Understand how each subsequent lesson fits into the class objectives.
\goal Do an initial evaluation of how relationship-focused your religion is.
\end{goals}

\begin{intro}
This class is about building a genuine and deep relationship with God.  Perhaps it's obvious, but the Christians life is more than simply `making it to heaven'.  After all, God must have a reason for creating man.  It is possible for us to know the facts of the Bible, to know the accounts of Abraham, Moses and Jesus, to be able to recite the plan of salvation, but forget that the reason for all of it is that God wants relationship with people.  It is even possible for us to live by the moral principles of the Bible and yet forget that living righteously is not an end in itself, but rather a means by which we improve our relationship with God.  God has always desired a people of His own, who choose Him voluntarily.  No passage brings this theme out better than Jeremiah 31:31-33.  No other Old Testament passage gives us such insight into what the New Covenant was supposed to be and how it was different that what came before.  Perhaps we just go through the motions in Christianity.  Is God far away, or is he close?  Sometimes what we are interested in is just do good, don't get too close to God.  After all, He's far away in heaven.  If I can just do what's right here on the earth, then I'll go to heaven and everything will be fine.  The possibility that I might have a real, genuine relationship with God seems out of reach.  He doesn't talk to me, and when I talk to Him I don't get an answer back.  
\end{intro}

\begin{bible}
\bv{Jeremiah}{(31:31-33)}

See page \pageref{sec:KeyPassage}

\bve{Hebrews}{(8:1-13)}

1 Now the point in what we are saying is this: we have such a high priest, one who is seated at the right hand of the throne of the Majesty in heaven, 2 a minister in the holy places, in the true tent that the Lord set up, not man. 3 For every high priest is appointed to offer gifts and sacrifices; thus it is necessary for this priest also to have something to offer. 4 Now if he were on earth, he would not be a priest at all, since there are priests who offer gifts according to the law. 5 They serve a copy and shadow of the heavenly things. For when Moses was about to erect the tent, he was instructed by God, saying, ``See that you make everything according to the pattern that was shown you on the mountain.'' 6 But as it is, Christ has obtained a ministry that is as much more excellent than the old as the covenant he mediates is better, since it is enacted on better promises. 7 For if that first covenant had been faultless, there would have been no occasion to look for a second. 8 For he finds fault with them when he says: ``Behold, the days are coming, declares the Lord, when I will establish a new covenant with the house of Israel and with the house of Judah, 9 not like the covenant that I made with their fathers on the day when I took them by the hand to bring them out of the land of Egypt. For they did not continue in my covenant, and so I showed no concern for them, declares the Lord. 10 For this is the covenant that I will make with the house of Israel after those days, declares the Lord: I will put my laws into their minds, and write them on their hearts, and I will be their God, and they shall be my people. 11 And they shall not teach, each one his neighbor and each one his brother, saying, `Know the Lord,' for they shall all know me, from the least of them to the greatest. 12 For I will be merciful toward their iniquities, and I will remember their sins no more.'' 13 In speaking of a new covenant, he makes the first one obsolete. And what is becoming obsolete and growing old is ready to vanish away.

\bvh{Exodus}{(6:1-9)}

1 But the Lord replied to Moses, ``Now you are going to see what I will do to Pharaoh: he will let them go because of My strong hand; he will drive them out of his land because of My strong hand.''

2 Then God spoke to Moses, telling him, ``I am Yahweh.  3 I appeared to Abraham, Isaac, and Jacob as God Almighty, but I did not reveal My name Yahweh to them.  4 I also established My covenant with them to give them the land of Canaan, the land they lived in as foreigners.  5 Furthermore, I have heard the groaning of the Israelites, whom the Egyptians are forcing to work as slaves, and I have remembered My covenant.

6 ``Therefore tell the Israelites: I am Yahweh, and I will deliver you from the forced labor of the Egyptians and free you from slavery to them. I will redeem you with an outstretched arm and great acts of judgment.  7 I will take you as My people, and I will be your God. You will know that I am Yahweh your God, who delivered you from the forced labor of the Egyptians.  8 I will bring you to the land that I swore to give to Abraham, Isaac, and Jacob, and I will give it to you as a possession. I am Yahweh.''  9 Moses told this to the Israelites, but they did not listen to him because of their broken spirit and hard labor.

\bvh{Leviticus}{(22:31-33)}

31 ``You are to keep My commands and do them; I am Yahweh.  32 You must not profane My holy name; I must be treated as holy among the Israelites. I am Yahweh who sets you apart,  33 the One who brought you out of the land of Egypt to be your God; I am Yahweh.''

\bve{John}{(3:16)}

16 ``For God so loved the world, that he gave his only Son, that whoever believes in him should not perish but have eternal life.''

\end{bible}

\begin{discussion}

\dsubsec{Introduction}{900-905}{5}

Don't be too familiar with God.  The `buddy' Jesus is pervasive, but let's not throw the baby out with the bathwater.

\dsubsec{Class Theme, Passage, and Objectives}{905-920}{15}



\dsubsec{Lesson Overview}{920-935}{15}

Each lesson contains the learning objectives for that class, the scriptures that we will be covering during class.  And, a list of questions to answer \emph{prior} to coming to class.

\begin{description}

\item[02-I Will Be Their God] God created the world and the people in it and showed Himself to them.

\item[03-They Shall Be My People] God wants a people for Himself.

\item[04-The Covenant I Made With Their Fathers] Israel was God's chosen people under the Old Covenant.

\item[05-I Was Their Husband] 

\item[06-My Covenant They Broke]

\item[07-Behold The Days Are Coming]

\item[08-I Will Make A New Covenant]

\item[09-Written On Their Hearts]

\item[10-For They Shall All Know Me]

\item[11-From The Least To The Greatest]

\item[12-I Will Forgive Their Iniquity]

\item[13-Review]


\end{description}

\dsubsec{Personal Evaluation}{935-940}{5}

Key Questions to ask yourself through this course:

\begin{itemize}
\item Is a relationship really what God is looking for from me?
\item Is a relationship really what I'm looking for from God?
\item What can I do to grow my relationship with God?
\item How will I know that I'm closer to God?
\end{itemize}

\dsubsec{Review}{940-945}{5}

\end{discussion}

\begin{questions}
\q How do you know that God wants to have a relationship with you?
\q What does a relationship with God look like?
\q In what areas can I grow in my relationship with God?
\q How will I know that my relationship with God is improving?
\end{questions}