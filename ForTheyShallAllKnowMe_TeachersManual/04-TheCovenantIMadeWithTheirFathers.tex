\chapter{The Covenant I Made With Their Fathers}

\begin{goals}
\goal Examine the role that covenants play in God's relationship with man
\goal Establish the importance of faith in both the Old and New Covenants
\goal Reflect on any ways that Christians could `return' to the weaknesses of Old Covenant
\end{goals}

\begin{bible}

\bvh{Genesis}{(6:17-18)}

17 ``Understand that I am bringing a flood —- floodwaters on the earth to destroy every creature under heaven with the breath of life in it. Everything on earth will die.  18 But I will establish My covenant with you, and you will enter the ark with your sons, your wife, and your sons' wives. 

\bvh{Galatians}{(3:1-29)}
3 You foolish Galatians! Who has hypnotized you, before whose eyes Jesus Christ was vividly portrayed as crucified?  2 I only want to learn this from you: Did you receive the Spirit by the works of the law or by hearing with faith?  3 Are you so foolish? After beginning with the Spirit, are you now going to be made complete by the flesh?  4 Did you suffer so much for nothing—if in fact it was for nothing?  5 So then, does God supply you with the Spirit and work miracles among you by the works of the law or by hearing with faith?

6 Just as Abraham believed God, and it was credited to him for righteousness,  7 then understand that those who have faith are Abraham's sons.  8 Now the Scripture saw in advance that God would justify the Gentiles by faith and told the good news ahead of time to Abraham, saying, All the nations will be blessed through you.  9 So those who have faith are blessed with Abraham, who had faith.

10 For all who rely on the works of the law are under a curse, because it is written: Everyone who does not continue doing everything written in the book of the law is cursed.  11 Now it is clear that no one is justified before God by the law, because the righteous will live by faith.  12 But the law is not based on faith; instead, the one who does these things will live by them.  13 Christ has redeemed us from the curse of the law by becoming a curse for us, because it is written: Everyone who is hung on a tree is cursed.  14 The purpose was that the blessing of Abraham would come to the Gentiles by Christ Jesus, so that we could receive the promised Spirit through faith.

15 Brothers, I'm using a human illustration. No one sets aside or makes additions to even a human covenant that has been ratified.  16 Now the promises were spoken to Abraham and to his seed. He does not say ``and to seeds,'' as though referring to many, but referring to one, and to your seed, who is Christ.  17 And I say this: The law, which came 430 years later, does not revoke a covenant that was previously ratified by God and cancel the promise.  18 For if the inheritance is from the law, it is no longer from the promise; but God granted it to Abraham through the promise.

19 Why then was the law given? It was added because of transgressions until the Seed to whom the promise was made would come. The law was put into effect through angels by means of a mediator.  20 Now a mediator is not for just one person, but God is one.  21 Is the law therefore contrary to God’s promises? Absolutely not! For if a law had been given that was able to give life, then righteousness would certainly be by the law.  22 But the Scripture has imprisoned everything under sin's power, so that the promise by faith in Jesus Christ might be given to those who believe.  23 Before this faith came, we were confined under the law, imprisoned until the coming faith was revealed.  24 The law, then, was our guardian until Christ, so that we could be justified by faith.  25 But since that faith has come, we are no longer under a guardian,  26 for you are all sons of God through faith in Christ Jesus.

27 For as many of you as have been baptized into Christ have put on Christ like a garment.  28 There is no Jew or Greek, slave or free, male or female; for you are all one in Christ Jesus.  29 And if you belong to Christ, then you are Abraham’s seed, heirs according to the promise. 

\end{bible}

\begin{discussion}

\dsubsec{Introduction}{900-905}{5}

\dsubsec{Examine the role that covenants play in God's relationship with man}{905-915}{10}

Genesis 6:17-18  God made a covenant with Noah

(Genesis 17:1-8) God made a Covenant with Abraham

Leviticus 26:1-45 God made a covenant with Israel

% This is actually used in lesson 08
Luke 22:14-23 Jesus' blood provides our New Covenant.  Same verse that Paul quotes in I Corinthians that we read for the Lord's Supper.

% probably need the Hebrews passage that says covenants are established with blood.

\dsubsec{Establish the importance of faith in both the Old and New Covenants}{915-925}{10}

Galatians 3:1-29 Even the Old Covenant was faith-based

Hebrews 11 The Hall of faith.

\dsubsec{Reflect on any ways that Christians could `return' to the Old Covenant.}{925-940}{15}

Back to Galatians.  Examine what it was that they were doing to 

\dsubsec{Review}{940-945}{5}
\end{discussion}

\begin{questions}
\q What is a covenant?
\q List all the covenants God has made with man through time.
\q Covenants usually have responsibilities for both parties involved.  For each covenant you listed in question 2, list God's responsibilities and man responsibilities.
\q Starting from Galatians 3:10-12 answer the following:  What role did faith play under the Old Covenant?
\end{questions}