\chapter{I Will Make a New Covenant}

\begin{goals}
\goal Understand what the New Covenant is
\goal Examine why the New Covenant was needed
\goal Assess how God's relationship with His people changed from the Old to the New Covenants
\end{goals}

\begin{bible}

\bve{Luke}{(22:14-23)}

14 And when the hour came, he reclined at table, and the apostles with him. 15 And he said to them, ``I have earnestly desired to eat this Passover with you before I suffer. 16 For I tell you I will not eat it until it is fulfilled in the kingdom of God.'' 17 And he took a cup, and when he had given thanks he said, ``Take this, and divide it among yourselves. 18 For I tell you that from now on I will not drink of the fruit of the vine until the kingdom of God comes.'' 19 And he took bread, and when he had given thanks, he broke it and gave it to them, saying, ``This is my body, which is given for you. Do this in remembrance of me.'' 20 And likewise the cup after they had eaten, saying, ``This cup that is poured out for you is the new covenant in my blood. 21 But behold, the hand of him who betrays me is with me on the table. 22 For the Son of Man goes as it has been determined, but woe to that man by whom he is betrayed!'' 23 And they began to question one another, which of them it could be who was going to do this.

\bvh{Ephesians}{(2:1-21)}

And you were dead in your trespasses and sins 2 in which you previously walked according to the ways of this world, according to the ruler who exercises authority over the lower heavens, the spirit now working in the disobedient. 3 We too all previously lived among them in our fleshly desires, carrying out the inclinations of our flesh and thoughts, and we were by nature children under wrath as the others were also. 4 But God, who is rich in mercy, because of His great love that He had for us, 5 made us alive with the Messiah even though we were dead in trespasses. You are saved by grace! 6 Together with Christ Jesus He also raised us up and seated us in the heavens, 7 so that in the coming ages He might display the immeasurable riches of His grace through His kindness to us in Christ Jesus. 8 For you are saved by grace through faith, and this is not from yourselves; it is God's gift-- 9 not from works, so that no one can boast. 10 For we are His creation, created in Christ Jesus for good works, which God prepared ahead of time so that we should walk in them.

11 So then, remember that at one time you were Gentiles in the flesh--called ``the uncircumcised'' by those called ``the circumcised,'' which is done in the flesh by human hands. 12 At that time you were without the Messiah, excluded from the citizenship of Israel, and foreigners to the covenants of the promise, without hope and without God in the world. 13 But now in Christ Jesus, you who were far away have been brought near by the blood of the Messiah. 14 For He is our peace, who made both groups one and tore down the dividing wall of hostility. In His flesh, 15 He made of no effect the law consisting of commands and expressed in regulations, so that He might create in Himself one new man from the two, resulting in peace. 16 He did this so that He might reconcile both to God in one body through the cross and put the hostility to death by it. 17 When the Messiah came, He proclaimed the good news of peace to you who were far away and peace to those who were near. 18 For through Him we both have access by one Spirit to the Father. 19 So then you are no longer foreigners and strangers, but fellow citizens with the saints, and members of God's household, 20 built on the foundation of the apostles and prophets, with Christ Jesus Himself as the cornerstone. 21 The whole building, being put together by Him, grows into a holy sanctuary in the Lord.

\bve{Romans}{(11:25-32)}

25 Lest you be wise in your own sight, I do not want you to be unaware of this mystery, brothers: a partial hardening has come upon Israel, until the fullness of the Gentiles has come in. 26 And in this way all Israel will be saved, as it is written,
\begin{quote}
``The Deliverer will come from Zion,\\
    he will banish ungodliness from Jacob'';\\
27 ``and this will be my covenant with them\\
    when I take away their sins.''
\end{quote}
28 As regards the gospel, they are enemies for your sake. But as regards election, they are beloved for the sake of their forefathers. 29 For the gifts and the calling of God are irrevocable. 30 For just as you were at one time disobedient to God but now have received mercy because of their disobedience, 31 so they too have now been disobedient in order that by the mercy shown to you they also may now receive mercy. 32 For God has consigned all to disobedience, that he may have mercy on all.

\bvh{Romans}{(3:19-25)}

19 Now we know that whatever the law says speaks to those who are subject to the law, so that every mouth may be shut and the whole world may become subject to God's judgment. 20 For no one will be justified in His sight by the works of the law, because the knowledge of sin comes through the law.

21 But now, apart from the law, God's righteousness has been revealed--attested by the Law and the Prophets 22 --that is, God's righteousness through faith in Jesus Christ, to all who believe, since there is no distinction. 23 For all have sinned and fall short of the glory of God. 24 They are justified freely by His grace through the redemption that is in Christ Jesus. 25 God presented Him as a propitiation through faith in His blood, to demonstrate His righteousness, because in His restraint God passed over the sins previously committed.

\bvh{Hebrews}{(10:1-18)}

Since the law has only a shadow of the good things to come, and not the actual form of those realities, it can never perfect the worshipers by the same sacrifices they continually offer year after year. 2 Otherwise, wouldn't they have stopped being offered, since the worshipers, once purified, would no longer have any consciousness of sins? 3 But in the sacrifices there is a reminder of sins every year. 4 For it is impossible for the blood of bulls and goats to take away sins.

5 Therefore, as He was coming into the world, He said:
\begin{quote}
You did not want sacrifice and offering,
but You prepared a body for Me.
6 You did not delight
in whole burnt offerings and sin offerings.
7 Then I said, ``See--
it is written about Me
in the volume of the scroll--
I have come to do Your will, God!''
\end{quote}
8 After He says above, You did not want or delight in sacrifices and offerings, whole burnt offerings and sin offerings (which are offered according to the law), 9 He then says, See, I have come to do Your will. He takes away the first to establish the second. 10 By this will of God, we have been sanctified through the offering of the body of Jesus Christ once and for all.

11 Every priest stands day after day ministering and offering the same sacrifices time after time, which can never take away sins. 12 But this man, after offering one sacrifice for sins forever, sat down at the right hand of God. 13 He is now waiting until His enemies are made His footstool. 14 For by one offering He has perfected forever those who are sanctified. 15 The Holy Spirit also testifies to us about this. For after He says:
\begin{quote}
16 This is the covenant I will make with them
after those days, says the Lord:
I will put My laws on their hearts
and write them on their minds,
\end{quote}
17 He adds:
\begin{quote}
I will never again remember
their sins and their lawless acts.
\end{quote}
18 Now where there is forgiveness of these, there is no longer an offering for sin.

\bvh{IICorinthians}{(3:7-18)}

7 Now if the ministry of death, chiseled in letters on stones, came with glory, so that the Israelites were not able to look directly at Moses' face because of the glory from his face--a fading glory-- 8 how will the ministry of the Spirit not be more glorious? 9 For if the ministry of condemnation had glory, the ministry of righteousness overflows with even more glory. 10 In fact, what had been glorious is not glorious now by comparison because of the glory that surpasses it. 11 For if what was fading away was glorious, what endures will be even more glorious.

12 Therefore, having such a hope, we use great boldness. 13 We are not like Moses, who used to put a veil over his face so that the Israelites could not stare at the the end of what was fading away, 14 but their minds were closed. For to this day, at the reading of the old covenant, the same veil remains; it is not lifted, because it is set aside only in Christ. 15 Even to this day, whenever Moses is read, a veil lies over their hearts, 16 but whenever a person turns to the Lord, the veil is removed. 17 Now the Lord is the Spirit, and where the Spirit of the Lord is, there is freedom. 18 We all, with unveiled faces, are looking as in a mirror at the glory of the Lord and are being transformed into the same image from glory to glory; this is from the Lord who is the Spirit.
\end{bible}

\begin{discussion}

\dsubsec{Introduction}{900-905}{5}

\dsubsec{Understand what the New Covenant is}{905-915}{10}

\bvs{Luke}{(22:14-23)} Jesus' blood is the seal of the New Covenant

\bvs{Ephesians}{(2:1-21)} Jews and Gentiles reconciled to God through Christ.

\dsubsec{Examine why the New Covenant was needed}{915-925}{10}

\bvs{Romans}{(11:25-32)} God shut everyone up under sin, so He can show mercy.

\bvs{Romans}{(3:19-25)} Apart from the Law, God's grace is now available to everyone.

% TODO: This is the wrong verse for entering the Holy place.
%\bvs{Hebrews}{(10:1-23)} - Enter the Holy Place with God.
\bvs{Hebrews}{(10:1-23)} Jesus' sacrifice forgives and sanctifies, once for all.

\dsubsec{Assess how God's relationship with His people changed from the Old to the New Covenants}{925-940}{15}

\bvs{IICorinthians}{(3:7-18)} The veil has been removed.  So we know God directly.

\dsubsec{Review}{940-945}{5}
\end{discussion}

\begin{questions}
\q Describe in your own words.  The New Covenant is \ldots
\q According to Romans 11:25-32 what purpose did the hardening of the Israelites' hearts toward Jesus serve?
\q Using Hebrews 10, compare the effectiveness of animal sacrifices and the effectiveness of Jesus' sacrifice.
\q Using II Corinthians 3:7-18, tell what the differences are between our closeness to God under the new Covenant and the Israelites' closeness to god under the Old Covenant?
\end{questions}