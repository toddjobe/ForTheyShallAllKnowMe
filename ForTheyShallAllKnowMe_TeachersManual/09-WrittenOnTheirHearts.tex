\chapter{Written On Their Hearts}

\begin{goals}
\goal Know what your `heart' is and how the New Covenant is `written' upon it
\goal Determine some internal and external ways to evaluate our hearts
\goal Commit to continual heart transformation as a way to draw closer to God
\end{goals}

\begin{bible}
\bve{ISamuel}{(16:7)}
7 But the Lord said to Samuel, ``Do not look on his appearance or on the height of his stature, because I have rejected him. For the Lord sees not as man sees: man looks on the outward appearance, but the Lord looks on the heart.''

\bvh{IICorinthians}{(4:16-18)}
16 Therefore we do not give up. Even though our outer person is being destroyed, our inner person is being renewed day by day. 17 For our momentary light affliction is producing for us an absolutely incomparable eternal weight of glory. 18 So we do not focus on what is seen, but on what is unseen. For what is seen is temporary, but what is unseen is eternal.

\bve{Ezra}{(7:10)}
For Ezra had set his heart to study the Law of the Lord, and to do it and to teach his statutes and rules in Israel.

\bve{Hebrews}{(4:1-13)}
Therefore, while the promise of entering his rest still stands, let us fear lest any of you should seem to have failed to reach it. 2 For good news came to us just as to them, but the message they heard did not benefit them, because they were not united by faith with those who listened. 3 For we who have believed enter that rest, as he has said,
\begin{quote}
``As I swore in my wrath,`They shall not enter my rest,'\thinspace''
\end{quote}
although his works were finished from the foundation of the world. 4 For he has somewhere spoken of the seventh day in this way: ``And God rested on the seventh day from all his works.'' 5 And again in this passage he said,
\begin{quote}
``They shall not enter my rest.''
\end{quote}
6 Since therefore it remains for some to enter it, and those who formerly received the good news failed to enter because of disobedience, 7 again he appoints a certain day, ``Today,'' saying through David so long afterward, in the words already quoted,
\begin{quote}
``Today, if you hear his voice, do not harden your hearts.''
\end{quote}
8 For if Joshua had given them rest, God would not have spoken of another day later on. 9 So then, there remains a Sabbath rest for the people of God, 10 for whoever has entered God's rest has also rested from his works as God did from his.

11 Let us therefore strive to enter that rest, so that no one may fall by the same sort of disobedience. 12 For the word of God is living and active, sharper than any two-edged sword, piercing to the division of soul and of spirit, of joints and of marrow, and discerning the thoughts and intentions of the heart. 13 And no creature is hidden from his sight, but all are naked and exposed to the eyes of him to whom we must give account.

\bve{IJohn}{(3:19-24)}
19 By this we shall know that we are of the truth and reassure our heart before him; 20 for whenever our heart condemns us, God is greater than our heart, and he knows everything. 21 Beloved, if our heart does not condemn us, we have confidence before God; 22 and whatever we ask we receive from him, because we keep his commandments and do what pleases him. 23 And this is his commandment, that we believe in the name of his Son Jesus Christ and love one another, just as he has commanded us. 24 Whoever keeps his commandments abides in God, and God in him. And by this we know that he abides in us, by the Spirit whom he has given us.

\bvh{Romans}{(12:1-2)}
1 Therefore, brothers, by the mercies of God, I urge you to present your bodies as a living sacrifice, holy and pleasing to God; this is your spiritual worship. 2 Do not be conformed to this age, but be transformed by the renewing of your mind, so that you may discern what is the good, pleasing, and perfect will of God.

\end{bible}

\begin{discussion}
\dsubsec{Introduction}{900-905}{5}

Written on their hearts as opposed to simply written on tablets of stone.

The heart is `soft' as opposed to `hard'.  Appreciate in what ways our heart can be measured.

\dsubsec{Know what your `heart' is and how the New Covenant is `written' upon it}{905-915}{10}

\bvs{ISamuel}{(16:7)} Man looks at the outward appearance.  God looks at the heart

\bvs{IICorinthians}{(4:16-18)} The inner part of us that can be renewed

\bvs{Ezra}{(7:10)} Ezra chose what His heart would be like.

\bvs{Deuteronomy}{(10:12-22)} Lesson 3.  God demands our hearts.

\dsubsec{Appreciate the internal and external ways to evaluate our hearts}{915-925}{10}

\bvs{Hebrews}{(4:1-13)} Don't have the Israelites' heart, but have faith guided by God's Word.

And, it's supposed not only to govern our outward behavior, but also how we feel and think.

\bvs{IJohn}{(3:19-24)} A heart that has love and obedience gives us confidence.\\

\dsubsec{Commit to continual heart transformation as a way to draw closer to God}{925-940}{15}

\bvs{Ezra}{(7:10)} We can choose which way it will go

\bvs{Romans}{(12:1-2)} Christians commit to a total life transformation.

\bvs{Ephesians}{(3:14-18)} Our hearts is strengthened when we understand God's love

% This is a candidate for moving
\bvs{Philippians}{(3:3-10)} We are the circumcision, and focus our lives on knowing Christ.

\dsubsec{Review}{940-945}{5}
\end{discussion}

\begin{questions}
\q What is your heart as it's used in the Bible?
\q List some of the ways that Hebrews 4:1-13 teaches us to avoid having a heart like the Israelites.
\q Think of a theme, intimately tied to our hearts, that connect Deuteronomy 10:12-22, and I John 3:19-24.
\q Think of a few practical ways that you can change your own heart in order to grow in your relationship with God.
\end{questions}