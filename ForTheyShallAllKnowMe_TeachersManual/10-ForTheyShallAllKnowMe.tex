\chapter{For They Shall All Know Me}

\begin{goals}
\goal Show that God's people are now joined by a common faith not a common ancestry 
\goal Elaborate on what it means to `know' God
\goal Come up with a plan to know God better
\end{goals}

\begin{bible}
\bve{Romans}{(4:1-25)}
What then shall we say was gained by Abraham, our forefather according to the flesh? 2 For if Abraham was justified by works, he has something to boast about, but not before God. 3 For what does the Scripture say? ``Abraham believed God, and it was counted to him as righteousness.'' 4 Now to the one who works, his wages are not counted as a gift but as his due. 5 And to the one who does not work but believes in him who justifies the ungodly, his faith is counted as righteousness, 6 just as David also speaks of the blessing of the one to whom God counts righteousness apart from works:
\begin{quote}
7 ``Blessed are those whose lawless deeds are forgiven, and whose sins are covered;
8 blessed is the man against whom the Lord will not count his sin.''
\end{quote}
9 Is this blessing then only for the circumcised, or also for the uncircumcised? For we say that faith was counted to Abraham as righteousness. 10 How then was it counted to him? Was it before or after he had been circumcised? It was not after, but before he was circumcised. 11 He received the sign of circumcision as a seal of the righteousness that he had by faith while he was still uncircumcised. The purpose was to make him the father of all who believe without being circumcised, so that righteousness would be counted to them as well, 12 and to make him the father of the circumcised who are not merely circumcised but who also walk in the footsteps of the faith that our father Abraham had before he was circumcised.

13 For the promise to Abraham and his offspring that he would be heir of the world did not come through the law but through the righteousness of faith. 14 For if it is the adherents of the law who are to be the heirs, faith is null and the promise is void. 15 For the law brings wrath, but where there is no law there is no transgression.

16 That is why it depends on faith, in order that the promise may rest on grace and be guaranteed to all his offspring—not only to the adherent of the law but also to the one who shares the faith of Abraham, who is the father of us all, 17 as it is written, ``I have made you the father of many nations''—in the presence of the God in whom he believed, who gives life to the dead and calls into existence the things that do not exist. 18 In hope he believed against hope, that he should become the father of many nations, as he had been told, ``So shall your offspring be.'' 19 He did not weaken in faith when he considered his own body, which was as good as dead (since he was about a hundred years old), or when he considered the barrenness of Sarah's womb. 20 No unbelief made him waver concerning the promise of God, but he grew strong in his faith as he gave glory to God, 21 fully convinced that God was able to do what he had promised. 22 That is why his faith was ``counted to him as righteousness.'' 23 But the words ``it was counted to him'' were not written for his sake alone, 24 but for ours also. It will be counted to us who believe in him who raised from the dead Jesus our Lord, 25 who was delivered up for our trespasses and raised for our justification.

\bvh{Philippians}{(3:3-10)}
1 For we are the circumcision, the ones who serve by the Spirit of God, boast in Christ Jesus, and do not put confidence in the flesh-- 4 although I once also had confidence in the flesh. If anyone else thinks he has grounds for confidence in the flesh, I have more: 5 circumcised the eighth day; of the nation of Israel, of the tribe of Benjamin, a Hebrew born of Hebrews; regarding the law, a Pharisee; 6 regarding zeal, persecuting the church; regarding the righteousness that is in the law, blameless.

7 But everything that was a gain to me, I have considered to be a loss because of Christ. 8 More than that, I also consider everything to be a loss in view of the surpassing value of knowing Christ Jesus my Lord. Because of Him I have suffered the loss of all things and consider them filth, so that I may gain Christ 9 and be found in Him, not having a righteousness of my own from the law, but one that is through faith in Christ--the righteousness from God based on faith. 10 My goal is to know Him and the power of His resurrection and the fellowship of His sufferings, being conformed to His death,

\bve{Ephesians}{(3:14-18)}
14 For this reason I bow my knees before the Father, 15 from whom every family in heaven and on earth is named, 16 that according to the riches of his glory he may grant you to be strengthened with power through his Spirit in your inner being, 17 so that Christ may dwell in your hearts through faith—that you, being rooted and grounded in love, 18 may have strength to comprehend with all the saints what is the breadth and length and height and depth,

\bvh{Matthew}{(7:7-12)}
7 ``Keep asking, and it will be given to you. Keep searching, and you will find. Keep knocking, and the door will be opened to you. 8 For everyone who asks receives, and the one who searches finds, and to the one who knocks, the door will be opened. 9 What man among you, if his son asks him for bread, will give him a stone? 10 Or if he asks for a fish, will give him a snake? 11 If you then, who are evil, know how to give good gifts to your children, how much more will your Father in heaven give good things to those who ask Him! 12 Therefore, whatever you want others to do for you, do also the same for them--this is the Law and the Prophets.

\end{bible}

\begin{discussion}
\dsubsec{Introduction}{900-905}{5}

\dsubsec{Show that God's people are now joined by a common faith not a common ancestry }{905-915}{10}

\bvs{Romans}{(4:1-25)} Those who have the faith of Abraham are his children.

\bvs{Phil}{(3:3-10)} Lesson 9. also says we are the true circumcision.

\dsubsec{Elaborate on what it means to `know' God}{915-925}{10}

These passages emphasize the importance of knowing God, but not necessarily what there is to know!.
\bvs{Ephesians}{(3:14-18)} Lesson 9

\dsubsec{Come up with a plan to know God better}{925-940}{15}

\bvs{Matthew}{(7:7-12)} Keep asking and it will be given you.

\dsubsec{Review}{940-945}{5}
\end{discussion}

\begin{questions}
\q Thinking about Romans 4, interpret the phrase ``For they shall \emph{all} know me'' from Jeremiah 31:34
\q Explain the weakness of God's special people being based on physical genealogy.
\q Given that Christians are now God's special people, what should we focus on according to Philippians 3:3-10 and Ephesians 3:14-18.
\q List four practical you could try to know God better.
\end{questions}