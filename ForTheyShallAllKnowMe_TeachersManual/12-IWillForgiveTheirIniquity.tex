\chapter{I Will Forgive Their Iniquity}

\begin{goals}
\goal Know what sin is, why it separates us from God, and what God has done about it in the New Covenant
\goal Examine why it is easy for us to strive for perfect lawkeeping rather than faithfulness
\goal Consider ways to keep the guilt of past sins from damaging your current relationship with God
\end{goals}

\begin{bible}
\bve{Romans}{(3:5-31)}
5 But if our unrighteousness serves to show the righteousness of God, what shall we say? That God is unrighteous to inflict wrath on us? (I speak in a human way.) 6 By no means! For then how could God judge the world? 7 But if through my lie God's truth abounds to his glory, why am I still being condemned as a sinner? 8 And why not do evil that good may come?--as some people slanderously charge us with saying. Their condemnation is just.

9 What then? Are we Jews any better off? No, not at all. For we have already charged that all, both Jews and Greeks, are under sin, 10 as it is written:
\begin{quote}
``None is righteous, no, not one;
11     no one understands;
    no one seeks for God.
12 All have turned aside; together they have become worthless;
    no one does good,
    not even one.''
13 ``Their throat is an open grave;
    they use their tongues to deceive.''
``The venom of asps is under their lips.''
14     ``Their mouth is full of curses and bitterness.''
15 ``Their feet are swift to shed blood;
16     in their paths are ruin and misery,
17 and the way of peace they have not known.''
18     ``There is no fear of God before their eyes.''
\end{quote}
19 Now we know that whatever the law says it speaks to those who are under the law, so that every mouth may be stopped, and the whole world may be held accountable to God. 20 For by works of the law no human being will be justified in his sight, since through the law comes knowledge of sin.

21 But now the righteousness of God has been manifested apart from the law, although the Law and the Prophets bear witness to it-- 22 the righteousness of God through faith in Jesus Christ for all who believe. For there is no distinction: 23 for all have sinned and fall short of the glory of God, 24 and are justified by his grace as a gift, through the redemption that is in Christ Jesus, 25 whom God put forward as a propitiation by his blood, to be received by faith. This was to show God's righteousness, because in his divine forbearance he had passed over former sins. 26 It was to show his righteousness at the present time, so that he might be just and the justifier of the one who has faith in Jesus.

27 Then what becomes of our boasting? It is excluded. By what kind of law? By a law of works? No, but by the law of faith. 28 For we hold that one is justified by faith apart from works of the law. 29 Or is God the God of Jews only? Is he not the God of Gentiles also? Yes, of Gentiles also, 30 since God is one--who will justify the circumcised by faith and the uncircumcised through faith. 31 Do we then overthrow the law by this faith? By no means! On the contrary, we uphold the law.

\bve{Hebrews}{(9:15-28)}
15 Therefore he is the mediator of a new covenant, so that those who are called may receive the promised eternal inheritance, since a death has occurred that redeems them from the transgressions committed under the first covenant. 16 For where a will is involved, the death of the one who made it must be established. 17 For a will takes effect only at death, since it is not in force as long as the one who made it is alive. 18 Therefore not even the first covenant was inaugurated without blood. 19 For when every commandment of the law had been declared by Moses to all the people, he took the blood of calves and goats, with water and scarlet wool and hyssop, and sprinkled both the book itself and all the people, 20 saying, ``This is the blood of the covenant that God commanded for you.'' 21 And in the same way he sprinkled with the blood both the tent and all the vessels used in worship. 22 Indeed, under the law almost everything is purified with blood, and without the shedding of blood there is no forgiveness of sins.

23 Thus it was necessary for the copies of the heavenly things to be purified with these rites, but the heavenly things themselves with better sacrifices than these. 24 For Christ has entered, not into holy places made with hands, which are copies of the true things, but into heaven itself, now to appear in the presence of God on our behalf. 25 Nor was it to offer himself repeatedly, as the high priest enters the holy places every year with blood not his own, 26 for then he would have had to suffer repeatedly since the foundation of the world. But as it is, he has appeared once for all at the end of the ages to put away sin by the sacrifice of himself. 27 And just as it is appointed for man to die once, and after that comes judgment, 28 so Christ, having been offered once to bear the sins of many, will appear a second time, not to deal with sin but to save those who are eagerly waiting for him.

\bve{Romans}{(7:1-25)}
Or do you not know, brothers--for I am speaking to those who know the law--that the law is binding on a person only as long as he lives? 2 For a married woman is bound by law to her husband while he lives, but if her husband dies she is released from the law of marriage. 3 Accordingly, she will be called an adulteress if she lives with another man while her husband is alive. But if her husband dies, she is free from that law, and if she marries another man she is not an adulteress.

4 Likewise, my brothers, you also have died to the law through the body of Christ, so that you may belong to another, to him who has been raised from the dead, in order that we may bear fruit for God. 5 For while we were living in the flesh, our sinful passions, aroused by the law, were at work in our members to bear fruit for death. 6 But now we are released from the law, having died to that which held us captive, so that we serve in the new way of the Spirit and not in the old way of the written code.

7 What then shall we say? That the law is sin? By no means! Yet if it had not been for the law, I would not have known sin. For I would not have known what it is to covet if the law had not said, ``You shall not covet.'' 8 But sin, seizing an opportunity through the commandment, produced in me all kinds of covetousness. For apart from the law, sin lies dead. 9 I was once alive apart from the law, but when the commandment came, sin came alive and I died. 10 The very commandment that promised life proved to be death to me. 11 For sin, seizing an opportunity through the commandment, deceived me and through it killed me. 12 So the law is holy, and the commandment is holy and righteous and good.

13 Did that which is good, then, bring death to me? By no means! It was sin, producing death in me through what is good, in order that sin might be shown to be sin, and through the commandment might become sinful beyond measure. 14 For we know that the law is spiritual, but I am of the flesh, sold under sin. 15 For I do not understand my own actions. For I do not do what I want, but I do the very thing I hate. 16 Now if I do what I do not want, I agree with the law, that it is good. 17 So now it is no longer I who do it, but sin that dwells within me. 18 For I know that nothing good dwells in me, that is, in my flesh. For I have the desire to do what is right, but not the ability to carry it out. 19 For I do not do the good I want, but the evil I do not want is what I keep on doing. 20 Now if I do what I do not want, it is no longer I who do it, but sin that dwells within me.

21 So I find it to be a law that when I want to do right, evil lies close at hand. 22 For I delight in the law of God, in my inner being, 23 but I see in my members another law waging war against the law of my mind and making me captive to the law of sin that dwells in my members. 24 Wretched man that I am! Who will deliver me from this body of death? 25 Thanks be to God through Jesus Christ our Lord! So then, I myself serve the law of God with my mind, but with my flesh I serve the law of sin.

\bvh{Romans}{(8:1-11)}
Therefore, no condemnation now exists for those in Christ Jesus, 2 because the Spirit's law of life in Christ Jesus has set you free from the law of sin and of death. 3 What the law could not do since it was limited by the flesh, God did. He condemned sin in the flesh by sending His own Son in flesh like ours under sin's domain, and as a sin offering, 4 in order that the law's requirement would be accomplished in us who do not walk according to the flesh but according to the Spirit. 5 For those who live according to the flesh think about the things of the flesh, but those who live according to the Spirit, about the things of the Spirit. 6 For the mind-set of the flesh is death, but the mind-set of the Spirit is life and peace. 7 For the mind-set of the flesh is hostile to God because it does not submit itself to God's law, for it is unable to do so. 8 Those who are in the flesh cannot please God. 9 You, however, are not in the flesh, but in the Spirit, since the Spirit of God lives in you. But if anyone does not have the Spirit of Christ, he does not belong to Him. 10 Now if Christ is in you, the body is dead because of sin, but the Spirit is life because of righteousness. 11 And if the Spirit of Him who raised Jesus from the dead lives in you, then He who raised Christ from the dead will also bring your mortal bodies to life through His Spirit who lives in you.

\end{bible}

\begin{discussion}
\dsubsec{Introduction}{900-905}{5}

Question about whether sin exists or not? Most people who don't believe in God, don't believe in sin either.
We are in the business of faithfulness, not perfect lawkeeping

\dsubsec{Know what sin is, why it separates us from God, and what God has done about it in the New Covenant}{905-915}{10}

\bvs{Romans}{(3:5-28)} Everyone is under sin, so that God may have mercy on everyone.

\bvs{Hebrews}{(9:15-28)} Jesus is the solution to Sin.

\dsubsec{Examine why it is easy for us to strive for perfect lawkeeping rather than faithfulness}{915-925}{10}

\bvs{Romans}{(7:1-25)} Paul illustrates the impossibility perfect law keeping.

\dsubsec{Consider ways to keep the guilt of past sins from damaging your current relationship with God}{925-940}{15}

\bvs{Romans}{(8:1-11)} There is no condemnation of sin for those in Christ Jesus.

\dsubsec{Review}{940-945}{5}
\end{discussion}

\begin{questions}
\q Summarize the message of Romans 3:5-21.
\q Using Romans 7 as a reference, explain why striving to be justified by perfect lawkeeping is bound to fail.
\q Summarize the message of Romans 8:1-11.
\q List three practical ways you can prevent the guilt of past sins from ruining your present and future.
\end{questions}